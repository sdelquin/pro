\documentclass{beamer}

\usepackage[utf8]{inputenc}
\usepackage[T1]{fontenc}
\usepackage[spanish]{babel}
\usepackage{amsmath}
\usepackage{pifont}
\usepackage{hyperref}

\setbeamertemplate{navigation symbols}{}

\usetheme{Berlin}
\usefonttheme{professionalfonts}

\title{Programación}
\subtitle{Desarrollo de Aplicaciones Web}
\author{Sergio Delgado Quintero}
\institute{IES Puerto de la Cruz - Telesforo Bravo}
\date{\today}

\begin{document}

\begin{frame}
    \titlepage
\end{frame}

\begin{frame}{Tabla de Contenidos}
    \tableofcontents
\end{frame}

\section{El profe}
\begin{frame}{Algo sobre mi}
    \begin{itemize}
        \item Sergio Delgado Quintero.
        \item Estudié \textbf{Ingeniería Informática} en la \textit{Universidad de La Laguna}.
        \item Catedrático de Enseñanza Secundaria especialidad en Informática.
        \item Casi 20 años programando (y enseñando) \textbf{Python}.
        \item Cofundador de \href{https://pythoncanarias.es}{Python Canarias}.
        \item Muchos proyectos desarrollados.
        \item Contacto: \texttt{sdelqui@gobiernodecanarias.org}
    \end{itemize}
\end{frame}

\section{El módulo}
\begin{frame}{Módulo: Programación}
    \begin{itemize}
        \item Primer curso del CFGS Desarrollo de Aplicaciones Web.
        \item 6 horas semanales.
        \item Lenguaje de programación: \textbf{Python}.
        \item Aula virtual CAMPUS: \href{https://peq.es/pro}{peq.es/pro}
        \item Contenidos: \href{https://peq.es/python}{peq.es/python}
        \item Máquina virtual: \textbf{VirtualBox}.
    \end{itemize}
\end{frame}

\section{Sistema de evaluación}
\begin{frame}{Instrumentos de evaluación}
    \begin{itemize}
        \item Pruebas objetivas [PO] (\textbf{60\%})
        \begin{itemize}
            \item Pruebas objetivas teóricas [POT] (\textbf{25\%})
            \item Pruebas objetivas prácticas [POP] (\textbf{75\%})\footnote{Es necesario superar para realizar media.}
        \end{itemize}
        \item Tareas evaluables [TE] (\textbf{40\%})
        \begin{itemize}
            \item Tareas evaluables: Ejercicios [TEE] (\textbf{50\%})
            \item Tareas evaluables: Proyectos [TEP] (\textbf{50\%})\footnote{Se realizará por grupos.}
        \end{itemize}
    \end{itemize}

    \begin{block}{\tiny{Cálculo de la nota}}
        \vspace{-1.5em}
        \begin{align*}
        N_{ev} =\ & 0.60 * (0.25 * POT + 0.75 * POP) +\\
                & 0.40 * (0.5 * TEE + 0.5 * TEP)
        \end{align*}
    \end{block}
\end{frame}

\begin{frame}{Criterios de calificación}
    \begin{tabular}{|l|l|l|l|}
        \hline
        & 1ª Evaluación & 2ª Evaluación & 3ª Evaluación\\
        \hline
        1ª Evaluación & 100\% & - & -\\
        \hline
        2ª Evaluación & 30\%  & 70\% & -\\
        \hline
        3ª Evaluación & 20\%  & 20\% & 60\%\\
        \hline
    \end{tabular}\\[1em]

    \begin{block}{\tiny{Cálculo de la nota}}
        \vspace{-1.5em}
        \begin{align*}
            N_1 &= N_{ev}\\
            N_2 &= 0.3 * N_1 + 0.7 * N_{ev}\\
            N_3 &= 0.2 * N_1 + 0.2 * N_2 + 0.6 * N_{ev}
        \end{align*}
    \end{block}
\end{frame}

\begin{frame}{Resultados de aprendizaje}
    \framesubtitle{$[RA1-RA5]$}
    \begin{itemize}
        \item \textbf{RA1}: Reconoce la estructura de un programa informático, identificando y relacionando los elementos propios del lenguaje de programación utilizado.
        \item \textbf{RA2}: Escribe y prueba programas sencillos, reconociendo y aplicando los fundamentos de la programación orientada a objetos.
        \item \textbf{RA3}: Escribe y depura código, analizando y utilizando las estructuras de control del lenguaje.
        \item \textbf{RA4}: Desarrolla programas organizados en clases analizando y aplicando los principios de la programación orientada a objetos.
        \item \textbf{RA5}: Realiza operaciones de entrada y salida de información, utilizando procedimientos específicos del lenguaje y librerías de clases.
    \end{itemize}
\end{frame}

\begin{frame}{Resultados de aprendizaje}
    \framesubtitle{$[RA6-RA9]$}
    \begin{itemize}
        \item \textbf{RA6}: Escribe programas que manipulen información seleccionando y utilizando tipos avanzados de datos.
        \item \textbf{RA7}: Desarrolla programas aplicando características avanzadas de los lenguajes orientados a objetos y del entorno de programación.
        \item \textbf{RA8}: Utiliza bases de datos orientadas a objetos, analizando sus características y aplicando técnicas para mantener la persistencia de la información.
        \item \textbf{RA9}: Gestiona información almacenada en bases de datos manteniendo la integridad y consistencia de los datos.
    \end{itemize}
\end{frame}

\section{Unidades de trabajo}

\begin{frame}{Secuenciación}
    \begin{tabular}{|l|l|l|}
        \hline
        \textbf{Unidad} & \textbf{Nombre} & \textbf{Trimestre}\\
        \hline
        UT1 & Introducción a la programación & \textcolor{purple}{T1}\\
        \hline
        UT2 & Tipos de datos & \textcolor{purple}{T1}\\
        \hline
        UT3 & Control de flujo & \textcolor{purple}{T1}\\
        \hline
        \hline
        UT4 & Estructuras de datos I & \textcolor{teal}{T2}\\
        \hline
        UT5 & Estructuras de datos II & \textcolor{teal}{T2}\\
        \hline
        UT6 & Funciones & \textcolor{teal}{T2}\\
        \hline
        \hline
        UT7 & Programación orientada a objetos & \textcolor{orange}{T3}\\
        \hline
        UT8 & Librerías & \textcolor{orange}{T3}\\
        \hline
    \end{tabular}
\end{frame}

\begin{frame}{Relación UT/RA}
    \begin{tabular}{|l|l|l|l|l|l|l|l|l|}
        \hline
            & UT1  & UT2  & UT3  & UT4  & UT5   & UT6  & UT7   & UT8\\
        \hline
        RA1 & 10\% & 80\% & 10\% &      &       &      &       & \\
        \hline
        RA2 &      &      &      &      &       & 50\% & 50\%  & \\
        \hline
        RA3 &      &      & 90\% &      &       &      & 10\%  & \\
        \hline
        RA4 &      &      &      &      &       &      & 100\% & \\
        \hline
        RA5 &      &      &      &      & 100\% &      &       & \\
        \hline
        RA6 &      &      &      & 50\% & 40\%  &      &       &10\% \\
        \hline
        RA7 &      &      &      &      &       &      & 100\% & \\
        \hline
        RA8 &      &      &      &      &       &      &  & 100\% \\
        \hline
        RA9 &      &      &      &      &       &      &  & 100\% \\
        \hline
    \end{tabular}
\end{frame}

\begin{frame}{Entregables}
    \begin{tabular}{|l|l|l|l|l|}
        \hline
            & POT       & POP      & TEE       & TEP\\
        \hline
        UT1 & \ding{51} &          &           &\\
        \hline
        UT2 & \ding{51} &\ding{51} &           &\\
        \hline
        UT3 & \ding{51} &\ding{51} & \ding{51} & \ding{51}\\
        \hline
        UT4 & \ding{51} &\ding{51} &           & \\
        \hline
        UT5 & \ding{51} &\ding{51} &           & \\
        \hline
        UT6 & \ding{51} &\ding{51} & \ding{51} & \ding{51}\\
        \hline
        UT7 & \ding{51} &\ding{51} &           & \\
        \hline
        UT8 & \ding{51} &\ding{51} & \ding{51} & \\
        \hline
    \end{tabular}
\end{frame}

\section{Otros aspectos}

\begin{frame}{Material}
    \begin{columns}
        \begin{column}[t]{0.5\textwidth}
            \begin{block}{Material obligatorio}
                \begin{itemize}
                    \item Cuaderno.
                    \item Bolígrafo.
                \end{itemize}
            \end{block}
        \end{column}
        \begin{column}[t]{0.5\textwidth}
            \begin{block}{Material recomendado}
                \begin{itemize}
                    \item Disco duro externo USB.
                \end{itemize}
            \end{block}
        \end{column}
    \end{columns}
\end{frame}

\begin{frame}{Horario}
    \begin{tabular}{|l|l|l|l|l|l|}
        \hline
         & Lunes & Martes & Miércoles & Jueves & Viernes\\
        \hline        
        14:30 - 15:25 & & & & &\\
        \hline
        15:25 - 16:20 & & & & \textcolor{purple}{PRO} &\\
        \hline
        16:20 - 17:15 & & & & \textcolor{purple}{PRO} &\\
        \hline
        17:15 - 17:45 & {\tiny Recreo} & {\tiny Recreo} & {\tiny Recreo} & {\tiny Recreo} & {\tiny Recreo}\\
        \hline
        17:45 - 18:40 & & & & &\\
        \hline
        18:40 - 19:35 & & \textcolor{purple}{PRO} & \textcolor{purple}{PRO} & &\\
        \hline
        19:35 - 20:30 & & \textcolor{purple}{PRO} & \textcolor{purple}{PRO} & &\\
        \hline
    \end{tabular}
\end{frame}

\begin{frame}{IA}

    {\Huge IA significa Inteligencia Artificial}
\end{frame}

\begin{frame}{Algo sobre ti}
    \begin{enumerate}
        \item ¿Cómo te llamas? ¿Cómo quieres que te llamen?
        \item ¿En qué te has formado hasta ahora?
        \item ¿Por qué quieres estudiar DAW?
        \item ¿Cuál es tu hobby?
        \item ¿Cuál es tu artista/grupo favorito de música?
    \end{enumerate}
\end{frame}


\end{document}
